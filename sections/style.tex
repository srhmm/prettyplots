 \clearpage
\section{Colors and Style} 
\label{sec:style}
\subsection{Color Palettes} \hypertarget{cyclelists}{}
Here are some color schemes as a starting point. 
\index{Colors and Style!defining line colors (cycle lists)}
\index{Colors and Style!defining marker colors (scatter classes)}

\begin{figure}[h!]%   
	\begin{tikzpicture} 
	\def\file{data/lines.tsv} 
	\begin{groupplot}[ 
	group style={ group size=5 by 3,  horizontal sep=25pt,   vertical sep=30pt,
	},  xtick align=center,  ytick align=center, width=\textwidth
	]  
\pgfplotsset{grp-base/.style={
	 width=0.25\textwidth,
	 line width=1.5pt,
	 pretty nolabels, 
	 pretty spacy
	},  
} 
	\grouplines{\file}[grp-base, cycle list name=pr-colors, title={\footnotesize{\texttt{pr-colors}}}, ylabel={\texttt{prettyplots.sty}}]
	\grouplines{\file}[grp-base, cycle list name=pr-colors0, title={\footnotesize{\texttt{pr-colors0}}}]
	\grouplines{\file}[grp-base, cycle list name=pr-colors1, title={\footnotesize{\texttt{pr-colors1}}}]
	\grouplines{\file}[grp-base, cycle list name=pr-colors2, title={\footnotesize{\texttt{pr-colors2}}}]
	\grouplines{\file}[grp-base, cycle list name=pr-colors3, title={\footnotesize{\texttt{pr-colors3}}}]
	\\
	\groupconflines{data/lines_conf.tsv}[grp-base, title={\footnotesize{\texttt{pr-colors-conf}}} ]
	\groupconflines{data/lines_conf.tsv}[grp-base, title={\footnotesize{\texttt{pr-colors-conf1}}}, cycle list name=pr-colors-conf1]
	\grouplines{\file}[grp-base, title={\footnotesize{\texttt{pr-marks}}}, cycle list name=pr-marks][mark size=1pt]
	\grouplines{\file}[grp-base, title={\footnotesize{\texttt{pr-white-marks}}}, cycle list name=pr-white-marks][mark size=3pt]
	\groupconflines{data/lines_conf.tsv}[grp-base, title={\footnotesize{\texttt{pr-marks-conf}}}, cycle list name=pr-marks-conf][mark size=1pt]
	\\
	\grouplines{\file}[grp-base, cycle list name = vangogh3, title={\footnotesize{\texttt{vangogh3}}},ylabel={\texttt{palettes.sty}}]
	\grouplines{\file}[grp-base, cycle list name = hiroshige, title={\footnotesize{\texttt{hiroshige}}}]
	\grouplines{\file}[grp-base, cycle list name = monet, title={\footnotesize{\texttt{monet}}}]
	\grouplines{\file}[grp-base, cycle list name = pissaro, title={\footnotesize{\texttt{pissaro}}}]
	\grouplines{\file}[grp-base, cycle list name = mamba, title={\footnotesize{\texttt{mamba}}}] 
	]
	\end{groupplot} 
	\end{tikzpicture}  
	\caption{Cycle lists for line plots, use with \texttt{\tbs begin\{axis\}[cycle list name=$\langle$name$\rangle$]}.}
\end{figure}
\begin{figure}[h!]%   
	\begin{tikzpicture} 
	\def\file{data/classes.tsv} 
	\begin{groupplot}[ 
		group style={ group size=5 by 2,  horizontal sep=25pt,   vertical sep=30pt,
		},  xtick align=center,  ytick align=center, width=\textwidth
		]  
		\pgfplotsset{grp-base/.style={
				width=0.25\textwidth,
				line width=1.5pt,
				pretty nolabels, 
				pretty spacy
			},  
		} 
		\scatterGroupplot{\file}[grp-base, title={\footnotesize{\texttt{pr-scatter}}}, ylabel={\texttt{prettyplots.sty}}][pr-scatter]
		\scatterGroupplot{\file}[grp-base, title={\footnotesize{\texttt{pr-scatter-opacity}}} ][pr-scatter-opacity]
		\scatterGroupplot{\file}[grp-base, title={\footnotesize{\texttt{pr-scatter-analogous}}} ][pr-scatter-analogous]
		\scatterGroupplot{\file}[grp-base, title={\footnotesize{\texttt{pr-scatter-complementary}}} ][pr-scatter-complementary] 
		\scatterGroupplot{\file}[grp-base, title={\footnotesize{\texttt{pr-scatter-marks}}} ][pr-scatter-marks]  
		]
	\end{groupplot} 
\end{tikzpicture}  
	\caption{Scatter classes, use with \texttt{\tbs addplot+[$\langle$name$\rangle$]} in an \texttt{\tbs begin\{axis\}[pretty scatter]} environment.}
\end{figure}

\clearpage
\subsection{Custom Colors} 
To set your own colors manually, pass your color of choice to \tbs \texttt{addplot[black]}. You can also pre-define a list of colors that each call of \tbs \texttt{addplot} will use through in order. This is known as a  cycle list, which you can pass to the  \texttt{axis} keys as  \texttt{cycle list=$\langle$name$\rangle$}. 
\begin{figure}[h!]
\begin{minipage}{.65\textwidth} 
\begin{mintedtxtbox}[/examples/line\_colors.tex]
\begin{minted}{TeX} 
%% Create a cycle list defining the color of each line 
\pgfplotscreateplotcyclelist{pr-colors}{
    {pr-color1a}, 
    {pr-color1b}, 
    {pr-color1c}, 
    {pr-color1d}, 
    {pr-color1e}, 
    {pr-color1f}, 
    {pr-color1g}, 
    {pr-color1h} 
} 
%% Usage in an axis environment
\begin{axis}[pretty line, cycle list name=pr-colors]
\pgfplotsinvokeforeach{1,...,8}{  
    \addplot table[x=x, y=y#1] {data/lines.tsv};  } 
\end{axis}
		\end{minted}
	\end{mintedtxtbox}
\end{minipage}  
\begin{minipage}{.3\textwidth} 
	%% Create a cycle list defining the color of each line 
\pgfplotscreateplotcyclelist{pr-colors}{
	{pr-color1a}, 
	{pr-color1b}, 
	{pr-color1c}, 
	{pr-color1d}, 
	{pr-color1e}, 
	{pr-color1f}, 
	{pr-color1g}, 
	{pr-color1h} 
} 
%% Usage in an axis environment
\begin{tikzpicture}
	\begin{axis}[pretty line, cycle list name=pr-colors,width=\textwidth]
		\pgfplotsinvokeforeach{1,...,8}{  
			\addplot table[x=x, y=y#1] {data/lines.tsv};  } 
	\end{axis}
\end{tikzpicture}
%\lines{data/lines.tsv}[cycle list name = pr-colors, pretty nolabels, pretty spacy, width=\textwidth]
\end{minipage} 
\end{figure}
 \par 
You can change the above definition of \texttt{pr-colors} directly in \texttt{prettyplots.sty} to use consistent colors in all of your work. For experiments, you can also define method-specific colors as follows

\begin{figure}[h!]
	\begin{minipage}{.65\textwidth} 
	\begin{tcolorbox}
			\begin{minted}{TeX} 
%% Pick your colors
\colorlet{color-ourmethod}{pr-color1a}
\definecolor{color-baselineA}{HTML}{f3f4f6} 
\definecolor{color-baselineB}{HTML}{f3f4f6}   
%% Create a cycle list 
\pgfplotscreateplotcyclelist{colors-methods}{
    {color-ourmethod}, 
    {color-baselineA}, 
    {color-baselineB}
} 
%% Usage in an axis environment
\begin{axis}[pretty line, cycle list name=colors-methods]
    \pgfplotsinvokeforeach{1,...,3}{  
        \addplot table[x=x, y=y#1] {data/lines.tsv};
    } 
\end{axis}
			\end{minted}
		\end{tcolorbox}
	\end{minipage}  
	\begin{minipage}{.3\textwidth}  
		\colorlet{color-ourmethod}{pr-color1a}
		\definecolor{color-baselineA}{HTML}{d1d5db} 
		\definecolor{color-baselineB}{HTML}{d1d5db}    
		\pgfplotscreateplotcyclelist{colors-methods}{
			{color-ourmethod}, 
			{color-baselineA}, 
			{color-baselineB}
		} 
	\begin{tikzpicture}
		\begin{axis}[pretty line, cycle list name=colors-methods, pretty nolabels, pretty spacy, width=\textwidth]
		\pgfplotsinvokeforeach{1,...,3}{  
			\addplot table[x=x, y=y#1] {data/lines.tsv}; 
		} 
	\end{axis}
\end{tikzpicture}
	\end{minipage} 
\end{figure}

\clearpage 

\subsection{Line Style}
\index{Colors and Style!defining line style}
\index{Colors and Style!defining marker style}  

Similar to color, you can set up the line style for your plots as follows. 
\begin{figure}[h!]
	\begin{minipage}{.65\textwidth} 
		\begin{tcolorbox} 
			\begin{minted}{TeX} 
%% Create a cycle list that changes linestyle
\pgfplotscreateplotcyclelist{pr-linestyle}{
    {pr-color0a, solid}, 
    {pr-color0b, densely dashdotted}, 
    {pr-color0c, densely dotted},
    {pr-color0d, densely dashed}, 
    {pr-color0e, dashdotted},
    {pr-color0f, dashed}, 
    {pr-color0g, dotted},
    {pr-color0g!80, loosely dashdotted} 
}
\lines{data/lines.tsv}[cycle list name=pr-linestyle]
			\end{minted}
		\end{tcolorbox} 
\end{minipage} 
\begin{minipage}{.3\textwidth} 	
	\lines{data/lines.tsv}[cycle list name=pr-linestyle, pretty nolabels, width=\textwidth]	
\end{minipage} 
\end{figure}
\begin{figure}[h!]	
	\begin{minipage}{.65\textwidth} 	
		\begin{tcolorbox} 
		\begin{minted}{TeX} 
%% Create a cycle list that changes marker style
\pgfplotscreateplotcyclelist{pr-marks}{
    {pr-color1a,mark=*,
    mark options={pr-color1a,fill=pr-color1a}},  
    {pr-color1b,mark=triangle*,
    mark options={pr-color1b,fill=pr-color1b}},  
    {pr-color1c,mark=square*,
    mark options={pr-color1c,fill=pr-color1b}},  
    % ...      
}
\lines{data/lines.tsv}[cycle list name=pr-marks]
		\end{minted}
	\end{tcolorbox}
	\end{minipage} 
\begin{minipage}{.3\textwidth} 
\lines{data/lines.tsv}[cycle list name=pr-marks, pretty nolabels, width=\textwidth]
\end{minipage} 
\end{figure}
	
\pgfplotsset{pr-marks-base/.style={  
		mark size=2pt,
		mark options={fill opacity=1,line width=.8}
	},  
}
\clearpage
\subsection{Scatter Classes}

\index{Plot Types!scatter!with classes} 
For scatter plots, \texttt{pr-scatter} defines how each of the classes should appear and is passed to \texttt{addplot},
\begin{figure}[h!]
\begin{minipage}{.7\textwidth} 
	\begin{tcolorbox} 
		\begin{minted}{TeX}  
\begin{axis}[pretty scatter,pretty nolabels]
    \addplot [pr-scatter] table[x=x1,y=x2,meta=y,
    col sep=comma] {data/classes.csv};  
\end{axis}
		\end{minted}
	\end{tcolorbox}
\end{minipage} 
\begin{minipage}{.29\textwidth}  
	\begin{tikzpicture}
		\begin{axis}[
			pretty scatter,  
			xmin            = 0, 
			ymin            = 0, 
			ymax 			= 10, 
			xmax			= 10,
			legend style={at={(1.8,1.1)}, legend columns = 2}, 
			pretty nolabels, width=\textwidth
			]
			\addplot[pr-scatter]
			table[x = x,y = y,meta = label] {data/classes.tsv};
			\addplot[pr-scatter]
			table[x expr = \thisrow{x} + 0.5,y expr = \thisrow{y} +  1.25, meta = label2] {data/classes.tsv};
			%\legend{Class 0,Class 1,Class 2,Class 3,Class 4,Class 5,Class 6,Class 7,Class 8,Class 9,Class 10,Class 11,Class 12,Class 13}
		\end{axis}
	\end{tikzpicture}
\end{minipage} 	 
  
 
\end{figure}
 
 To replace  this with your own, 
\begin{figure}[h!]
	\begin{minipage}{.7\textwidth}  
\begin{mintedtxtbox}[/examples/scatter\_colors.tex] 
	\begin{minted}{TeX}
%% Define custom scatter colors
\pgfplotsset{ 
  my-scatter/.style={
    only marks, mark options={fill opacity=0.50},
    scatter src=explicit symbolic,
    scatter/classes={
        0={mark=*, pr-color1a, fill=pr-color1a!50},   
        1={mark=triangle*, pr-color1b, fill=pr-color1b!50},
        2={mark=diamond*, pr-color1c, fill=pr-color1c!50},
        3={mark=square*, pr-color1d, fill=pr-color1d!50}
        %no comma in the last line, otherwise latex complains
    }
  }
} 
\begin{axis}[pretty scatter,pretty nolabels]
    \addplot [my-scatter]
    table[x=x1,y=x2,meta=y, col sep=comma] {data/blobs.csv};  
\end{axis}
	\end{minted}
\end{mintedtxtbox}
	\end{minipage} 
	\begin{minipage}{.29\textwidth}  
		\pgfplotsset{ 
	my-scatter/.style={	 
		only marks, mark options={fill opacity=0.50},
		scatter src	=explicit symbolic,
		scatter/classes	= {
			0={mark=*, pr-color1a, fill=pr-color1a!50},   
			1={mark=triangle*, pr-color1b, fill=pr-color1b!50 },
			2={mark=diamond*, pr-color1c, fill=pr-color1c!50 },
			3={mark=square*, pr-color1d, fill=pr-color1d!50 } 
	}}
} 
\begin{tikzpicture}
	\begin{axis}[
		pretty scatter,   
		width= \textwidth,
		xmin            = 0, 
		ymin            = 0, 
		ymax 			= 10, 
		xmax			= 10,
		pretty nolabels, 
		legend style={at={(1.15,1)}}
		]
		\addplot [my-scatter]
		table[x = x1,y = x2,meta = y,col sep=comma] {data/blobs.csv};  
	\end{axis}
\end{tikzpicture}   
	\end{minipage} 
\end{figure}

  
  \clearpage
\index{Colors and Style!defining continuous colors (colormap)} 
\index{Plot Types!scatter!with colormap} 
Here us an example for cases where your label is continuous instead of discrete. 
\begin{figure}[h!]	
 		\begin{minipage}{0.65\textwidth}
	\begin{mintedtxtbox}[/data/blob.tsv]
		\begin{minted}{TeX}
%x    y    color value
x    y    col
-82.87    16.81    0.52
-75.32    44.21    0.41
\end{minted}
	\end{mintedtxtbox} 
\end{minipage}  
\end{figure}
\par
Then, you can pick a few transition colors and set up a color map as follows.
\begin{figure}[h!]		
	\begin{minipage}{0.65\textwidth}
		\begin{mintedtexbox}[/examples/scatter\_colormap\_custom.tex] 
			\inputminted{TeX}{lst/scatter_colormap_custom.txt} 
		\end{mintedtexbox} 
	\end{minipage}
	\begin{minipage}{0.3\textwidth}
			\begin{tikzpicture}
	\begin{axis}[
		pretty scatter colormap, pretty nolabels, 
		colormap={prmap1}{rgb255=(253,247,251) rgb255=(234,230,241) rgb255=(207,208,228) rgb255=(171,188,217) rgb255=(133,168,204) rgb255=(92,143,189) rgb255=(64,111,172) rgb255=(52,90,138) rgb255=(34,59,88)},  
		xmax=120, xmin =-120, ymax=100, ymin=-100,   width=\textwidth,   height=1.1\textwidth
		] 
		\addplot+[pr-scatter-meta]  table[meta=col] 
		{data/blob.tsv}; 
	\end{axis} 
\end{tikzpicture}  
	\end{minipage} 
	
\end{figure}
  
 