 	
%% Path labels

%\subsection{Labels}
\vspace{0.5cm}
\begin{figure}[h]
\begin{minipage}{0.45\textwidth}
\begin{tikzpicture}
    \begin{axis}[
    pretty line,
    pretty spacy, 
    example line, 
    no markers, thick, black,
    legend entries = {}]
    \addplot[smooth] table[x = x, y = y2] {data/line_odd.txt} 
    node[example label]  {strange line};
    \addplot[smooth] table[x = x, y = y3] {data/line_odd.txt} 
    node[example label] {odd line};
    \addplot table[x = x, y = y1] {data/line_odd.txt} 
    node[example label, inner sep = 4pt, pos = 0.3, rotate = 25] {weird line};
    \end{axis}
\end{tikzpicture}
\end{minipage}
\begin{minipage}{.55\textwidth}
	
    \inputminted{TeX}{lst/line_odd.txt}
	
\end{minipage}
\end{figure}

%% Annotating points

\vspace{2cm}
\begin{figure}[h]
	\begin{minipage}{0.45\textwidth}
		\begin{tikzpicture}
		\begin{axis}[
		pretty line marks,
		example marks, 
		legend style    = {at = {(1,.75)}, anchor = south east},
		xmax            = 10, 
		ymax            = 10, 
		xmin            = 0,
		]	
		\addplot table[x=x, y=y]  {data/marks.txt};	
		\addplot table[x=x, y expr=10-0.5*\thisrow{y}]  {data/marks.txt};
		%% annotations: 
		\node at (axis cs:6, 2.1)  {\scriptsize boring};
		\node at (axis cs:5.5, 2) (node1) {};
		\node at (axis cs:4.5, 0.5)  (node2) {};
		\draw [->, gray, thick] (node1) --(node2);
		\node[pin=-90:{\scriptsize special}] at (axis cs:9,5.8) {};	
		\end{axis}
		\end{tikzpicture} 
	\end{minipage}
	\begin{minipage}{0.55\textwidth}
		
		\inputminted{TeX}{lst/marks_boring.txt}
		
	\end{minipage}
\end{figure} 

%% Discrete, rotation + testing scale 10 labels, grid

%\subsection{Discrete labels, scale labels}
\vspace{.5cm}
\begin{figure}[h]
	\begin{minipage}{0.45\textwidth}
		\begin{tikzpicture}
			\begin{axis}[	
				example marks, 
				pretty line marks,
				pretty ygrid,
				pr-labels-tilted,
				scaled y ticks  = base 10:-2,
				symbolic x coords= {Germany,France,England,Saarland,Spain,Finland,USA},
				height          = 5cm,		
				legend entries  = {}, 
				ymax            = 1000
				]
				\addplot table[x=x, y expr=100*\thisrow{y1}] {data/countries.tsv};
			\end{axis}
		\end{tikzpicture}
	\end{minipage}
	\begin{minipage}{.55\textwidth}
		
		\inputminted{TeX}{lst/marks_quality2.txt}
		
	\end{minipage}
\end{figure}





%% Annotating points with visualization dependency

\vspace{0.5cm}
\begin{figure}[h]
	\begin{minipage}{.45\textwidth}
		\begin{tikzpicture}
		\begin{axis}[
		pretty scatter,
		example marks,
		legend entries = {},
		scatter/use mapped color = {draw=gray,fill=gray,fill opacity=0.5},
		ymin=0, ymax=10,
		xmin=0, xmax=10,
		height = 6cm, 
		width = 6cm
		]
		\addplot[only marks,
		nodes near coords*={\scriptsize $\pgfmathprintnumber \weight$},	
		every node near coord/.append style = {yshift = 1 pt},
		visualization depends on={\thisrow{weight} \as \weight}, 
		scatter/@pre marker code/.append style=
		{/tikz/mark size=0.5*\weight}
		] 
		table {
			x	y	weight
			2	2	4
			4	6	4
			6	4	6
			8	4	2
		};
		\end{axis}
		\end{tikzpicture}
	\end{minipage}
	\begin{minipage}{.55\textwidth}
		\vspace{.5cm}
		\inputminted{TeX}{lst/scatter_weighted.txt}
		
	\end{minipage}
	
	
	%% Labelled bars
	
	\begin{minipage}{0.45\textwidth}
		\vspace{3cm}
		\begin{tikzpicture}
		\begin{axis}[
		pretty bar, % small,
		pr-labels-tilted,
		bar width       = 0.5cm,
		cycle list name = prcl-ybar,
		height          = 6cm,
		width           = 6cm,
		ylabel          = {Rating}, 
		xlabel          = {}, 
		legend entries  = {},
		symbolic x coords= {Oslo,Berlin,Tokyo,Rio},
		%% label each bar with y value
		visualization depends on=y \as \y,
		nodes near coords ={\pgfmathprintnumber[precision=2]{\y}}, %TODO \y -> \pgfplotspointmeta? 
		%% label layout
		every node near coord/.append style={font=\scriptsize,color=black!80}, 
		ymin = 0]
		\addplot table[x=x, y=y3] {data/cities.tsv};
		\end{axis}
		\end{tikzpicture}	
	\end{minipage}
	\begin{minipage}{0.55\textwidth}	
		\vspace{3.75cm}
		\inputminted{TeX}{lst/bar_labelled.txt}
		
	\end{minipage}
\end{figure} 



%% Alternative: Point meta data

%\begin{figure}
%	\begin{tikzpicture}
%	\begin{axis}[
%		pretty scatter,
%		nodes near coords,
%		ymax=10,xmin=0,xmax=10,ymin=0,
%		width = 5cm, height = 4cm
%		]
%	\addplot[only marks, mark=*, point meta=explicit symbolic, c1] 
%	coordinates {
%		(2,2) [\scriptsize p1]
%		(3,7) [\scriptsize p2]
%		(6,4) [\scriptsize]
%	};
%	\end{axis}
%	\end{tikzpicture}
%\end{figure}
%


%% More heat

%\begin{figure}
%	\begin{minipage}{.55\textwidth}
%	\begin{tikzpicture}
%	\begin{axis}[
%	pretty heated,
%	example marks,
%	colorbar sampled,
%	scaled y ticks  = base 10:-2, 
%	scaled x ticks  = base 10:-2, 
%	colormap/bone,
%	legend entries  = {},
%	xmin=0, ymin=0,,xmax=1000, ymax=1000] 
%	\addplot [mark size = 2pt] table[x expr=\thisrow{x}*100, y expr=\thisrow{y}*100]  	{data/marks.txt};
%	\addplot [mark size = 2pt] table[x expr=\thisrow{x}*100, y expr=1000-\thisrow{y}*50]  {data/marks.txt};
%	\end{axis}
%	\end{tikzpicture}
%\end{minipage}
%\begin{minipage}{.45\textwidth}
%
%\inputminted{TeX}{lst/scatter_heated2.txt}
%
%\end{minipage}
%\end{figure}
\clearpage


%% Auxiliary lines


%\subsection{Error lines}
\vspace{1cm}
\begin{figure}[h]
	\begin{minipage}{0.45\textwidth}
		\begin{tikzpicture}
		\begin{axis}[
		pretty line, 
		example line, 
		%% variables:
		xtick           = {0,2,8,10}, 
		xticklabels     = {0, $t_0$,$t_1$,$t_2$ }, 
		xmax = 10, ymax = 10]
		%% error bars:
		\addplot[c1, mark=*] 
		plot[error bars/.cd,
		y dir           = minus, 
		y fixed relative= 1,
		x dir           = minus, 
		x fixed relative= 1,
		error mark      = none,
		error bar style	= {dotted}]
		coordinates {(0,0) (2,2) (8,8) (10, 10)};
		%% threshold line: 
		% \draw[gray, dotted] 
		% (axis cs:\pgfkeysvalueof{/pgfplots/xmin}, 5) -- (axis cs:\pgfkeysvalueof{/pgfplots/xmax}, 5);
		\end{axis}
		\end{tikzpicture}
	\end{minipage}
	\begin{minipage}{0.55\textwidth}
		
		\inputminted{TeX}{lst/line_xylines.txt}
		
	\end{minipage}
\end{figure}

\vspace{2cm}
\begin{figure}[h]
	\begin{minipage}{0.45\textwidth}
		\begin{tikzpicture}
		\begin{axis}[
		pretty line,  
		xtick           = {0,...,6}, 
		ytick           = {0,2,4,6,8,10}, 	 
		xmax = 6, ymax = 10,smooth,  ]
		\addplot+[dollarbill,  mark=*,error bars/.cd, y dir=both, y explicit] 
		coordinates { (0,6)
			(1,6) +=(0,0.5) -= (0,0.5)
			(2,5) +=(0,0.6) -= (0,0.6)
			(3,3) +=(0,0.3) -= (0,0.3)
			(4,2) +=(0,0.4) -= (0,0.4)
			(5,2) +=(0,0.3) -= (0,0.3)
			(6,1.8) 
		};	
		\end{axis}
		\end{tikzpicture}
	\end{minipage}
	\caption{Error line. \index{Plot Types!Line!with error bars}\index{Decorations!error marks}}
	\begin{minipage}{0.55\textwidth}
		\vspace{2.5cm}
		%TODO	\inputminted{TeX}{lst/line_confident.txt}
	\end{minipage}
\end{figure} 

%	\coordinate (a) at (axis cs:0,4.5);
%\coordinate (b) at (axis cs:0,5.5);
%\coordinate (c) at (rel axis cs:0,0);
%\coordinate (d) at (rel axis cs:1,0);
%\draw [gray,thin,sharp plot,dashed] (a -| c) -- (a -| d);
%\draw [gray,thin,sharp plot,dashed] (b -| c) -- (b -| d);



%% Fill between plots

%\subsection{Fill areas}
\index{Decorations!fill-between}
\vspace{1cm}
\begin{figure}[h]
	\begin{tikzpicture}
	\begin{axis}[
	pretty line,
	pretty ygrid, 
	example entries, 
	%major grid style= {gray,dotted}, 
	width           = .75*\textwidth, 
	height          = 5cm,
	ymax            = 1, 
	xmax            = 1
	]
	%% shading
	\addplot[forget plot, name path=lower, color=lightgray] % forget: no legend
	table[x index=0, y index=4, header=true] {data/line_shady.txt};
	\addplot[forget plot, name path=upper, color=lightgray] 
	table[x index=0, y index=5, header=true] {data/line_shady.txt};
	\addplot[forget plot, gainsboro, on layer=axis background] 
	fill between[of=lower and upper];
	\pgfplotsinvokeforeach{1,...,2}{
		\addplot[thick, c#1] 
		table[x index=0, y index=#1, header=true] {data/line_shady.txt};
		\addlegendentry{Shady Method #1} %TODO markers
	}
	\end{axis}
	\end{tikzpicture}
	\vspace{1cm}
	\inputminted{TeX}{lst/line_shady2.txt}
	
\end{figure}
\clearpage


%% Fill area

\begin{figure}[h]
	\begin{tikzpicture}
	\begin{axis}[
	pretty line, 
	example line,
	legend style = {at = {(0.85,0.6)}, inner xsep = 3pt}, 
	area legend,
	width = .6*\textwidth,
	ymax=1, xmax=1
	]
	\addplot[c2,fill=c2,opacity=0.3,smooth]
	table[x = x1, y = y1] {data/line_filled.txt}
	|- (axis cs:0,0) -- cycle;
	\addplot [c1,fill=c1,opacity=0.3,smooth]
	table[x = x1, y = y2] {data/line_filled.txt}
	|- (axis cs:0,0) -- cycle;
	%alternative to area legend:
	%\addlegendimage{only marks, mark=*, mark size=2pt,c1,opacity=0.3}
	%\addlegendimage{only marks, mark=*, mark size=2pt,c2,opacity=0.3}
	\end{axis}
	\end{tikzpicture}
	\vspace{1cm}
	\inputminted{TeX}{lst/line_shady1.txt}
	
\end{figure} 


%mixed legend
%\begin{figure}
%	\begin{tikzpicture}
%	\begin{axis}[
%		pretty line spacey, 
%		example entries,,
%		xmax=10, domain=0:10, samples=10,
%		]
%		\addplot {x^2};
%		\addplot[ybar,fill=mambacolor4,draw=mambacolor4,opacity=0.4,
%	ybar legend,mark=none,samples=5]
%	{-30*(x +4)};
%	\legend{first,second}
%	\end{axis}
%	\end{tikzpicture}
%\end{figure}


%\begin{figure}
%\begin{tikzpicture}
%\begin{axis}[legend pos=outer north east]
%\addplot3[surf,samples=9,domain=0:1]
%{(1-abs(2*(x-0.5))) * (1-abs(2*(y-0.5)))};
%\addlegendentry{$\phi_x \phi_y$}
%\addplot3+[ultra thick] coordinates {(0,0,0) (0.5,0,1) (1,0,0)};
%\addlegendentry{$\phi_x $}
%\addplot3+[ultra thick] coordinates {(1,0,0) (1,0.5,1) (1,1,0)};
%\addlegendentry{$\phi_y $}
%\end{axis}
%\end{tikzpicture}
%\end{figure}



%% Stacking two y axes 

%\subsection{Grouping}
\begin{figure}[h]
	\begin{minipage}{.45\textwidth}
		\begin{tikzpicture}
		% for small plots also use minipage st plots dont appear next to each other
		\begin{axis}[
		pretty line, 
		example line, 
		height          = 4.5cm,
		width           = 6.5cm,
		ylabel          = {ydomain 1},
		axis x line     = none,
		x axis line style={draw opacity=0},
		legend entries  ={},
		ymax            = 1]	
		\addplot[line width=1pt, c1,smooth] 
		table[x index=0, y expr=1-\thisrow{y2}, header=true] {data/line_odd.txt};
		\end{axis}
		\end{tikzpicture} 
		\begin{tikzpicture}
		\begin{axis}[
		pretty line, 
		example line,
		height          = 4.5cm,
		width           = 6.5cm,
		legend entries  ={},
		ylabel          = {ydomain 2}, 
		ymax            = 1,]	
		\addplot[line width=1pt, c1] 
		table[x index=0, y index=1, header=true] {data/line_odd.txt};
		\end{axis}
		\end{tikzpicture}
	\end{minipage}
	\begin{minipage}{0.55\textwidth}
		\vspace{1cm}
		\inputminted{TeX}{lst/line_yy.txt}
		
	\end{minipage}
\end{figure} 





%% Small ybar + groupplots

%\subsection{Grouping}
\begin{figure}[h]		
	\vspace{.5cm}
	\begin{minipage}{\textwidth}
		\vspace{.4cm}
		\begin{tikzpicture}
		\begin{groupplot}[
		pretty ybar small,
		pr-labels-tilted,
		cycle list name = prcl-ybar,
		%% Grouping
		group style     = {
			group size = 2 by 1, 
			horizontal sep = 15pt, 
			ylabels at = edge left,
			y descriptions at = edge left
		},	
		%% Attributes Group1
		height          = 4.25cm,
		symbolic x coords= {Germany,France,England,Saarland,Spain,Finland,USA},
		ylabel          = {Rating}, 
		xlabel          = {}, 
		legend entries  = { Living, Eating },
		legend style    = { at = {(1.69,1.2)},} ,
		ymin =0,ymax=10,
		%bar width illegal in groupplot, set in pretty ybar 
		]
		\nextgroupplot[width=6cm]
		\pgfplotsinvokeforeach{1,...,2}{ \addplot table[x=x, y=y#1] {data/countries.tsv}; }
		%% Attributes Group 2
		\nextgroupplot[
		width   = 4cm,
		symbolic x coords= {Oslo,Berlin,Tokyo,Rio},
		legend entries= {},
		ymin= 0,ymax=10 %mind that y has the same scale
		]
		\pgfplotsinvokeforeach{1,...,2}{ \addplot table[x=x, y=y#1] {data/cities.tsv}; }
		\end{groupplot}
		\end{tikzpicture}
		\vspace{1.5cm}
		\inputminted{TeX}{lst/bar_grouped.txt}
	\end{minipage} 
	\caption{\index{Plot Types!Bar}  \index{Axes!grouping}}
\end{figure} 


%% Stacked ybar
%\subsection{Stacking}
\begin{figure}[h]
	\vspace{.5cm}
	\begin{minipage}{.5\textwidth}
		\begin{tikzpicture}
		\begin{axis}[
		pretty ybar stacked,
		height          = 7cm,
		width           = 6cm,
		bar width       = .5cm,
		symbolic x coords= {y1,y2,y3,y4},
		cycle list name = prcl-stacked,
		ylabel          = {ylabel},
		%xlabel         = {xlabel},
		legend style    = {
			at      = {(0.5,-0.20)},
			anchor  = north,
			legend columns=-1},
		]
		\addplot plot coordinates {(y1,0) (y2,2) (y3,3) (y4,0)};
		\addplot plot coordinates {(y1,0) (y2,0) (y3,3) (y4,0)};
		\addplot plot coordinates {(y1,6) (y2,6) (y3,2) (y4,6)};
		\addplot plot coordinates {(y1,4) (y2,2) (y3,2)  (y4,4)};
		\legend{z1, z2, z3, z4}
		\end{axis}
		\end{tikzpicture}
	\end{minipage}
	\begin{minipage}{.5\textwidth}
		
		\inputminted{TeX}{lst/bar_ystack.txt}
		
	\end{minipage}
	
	
	
	%% Stacked xbar variant
	
	\begin{minipage}{.5\textwidth}	
		\vspace{3cm}
		\begin{tikzpicture}
		\begin{axis} [
		pretty xbar components,
		width           = 8cm,
		height          = 3cm,
		bar width       = 5pt,
		enlarge y limits= 0.04,
		xlabel          = {xlabel},
		ylabel          = {opt. ylabel},
		xtick           = {0, 0.25, 0.5, 0.75, 1},
		symbolic y coords={y4,y3,y2,y1},
		ytick           = {y4,y3,y2,y1},	
		]
		\addplot+[color=white,fill=gray] table[y index=0, x index=1, header=true] {data/components.tsv}; 
		\end{axis}
		\end{tikzpicture}
	\end{minipage}
	\begin{minipage}{.5\textwidth}
		\vspace{3cm}	
		\inputminted{TeX}{lst/bar_components.txt}
	\end{minipage}
\end{figure}

%% Interval ybar
%
%\begin{figure}
%	\begin{tikzpicture}
%	\begin{axis}[	
%	pretty ybar interval, 
%	xticklabel interval boundaries,
%	pretty discrete,
%	xlabel          = {xlabel},
%	ylabel          = {ylabel},
%	width           = 4cm, 
%	height          = 4cm, 
%	ymin            = 0,
%	]
%	\addplot[white, fill=gray] coordinates
%	{(0,2) (1,1) (3,2) (5,4) (6,3) (7,2) };
%	\end{axis}
%\end{tikzpicture}
%\end{figure}
%
%%% Interval xbar, no grid
%
%\begin{figure}
%\begin{tikzpicture}
%	\begin{axis}[	
%	pretty xbar interval,
%	ylabel          = {age group},
%	xlabel          = {something},
%	ytick           = data,%TODO yticks look weird, put mid bar 
%	xtick           = data,
%	yticklabel interval boundaries,
%	width           = 6cm, 
%	height          = 5cm, 
%	xmin            = 0,
%	xmax            = 10,
%%	xmajorgrids = true,
%%		major grid style   = white,
%	pretty labelshift, % after xbar interval and other pot. axis shifts
%]
%	\addplot[white, fill=gray]
%	coordinates {(2,0) (3,12) (6,18) (9,25) (7,45) (4,65) (3,80)};\end{axis}
%\end{tikzpicture}
%	\end{figure}



%% Stacked  xbar 
\index{Plot Types!xbar}
\vspace{1.5cm}
\begin{figure}[h]
	\begin{minipage}{0.5\textwidth}
		\begin{tcolorbox}[title=/stacked.tsv]
			\begin{minted}{TeX}
			%ys    x1 ...    xn
			y    x1 ...    xn
			y1    1.8
			...
			\end{minted} 
		\end{tcolorbox}
		\begin{tcolorbox}[title=/examples/stacked.tex]
			\inputminted{TeX}{lst/bar_xstack1.txt} 
		\end{tcolorbox}
	\end{minipage} 
	\begin{minipage}{0.5\textwidth}
		\begin{tikzpicture}
		\begin{axis}[	
		pretty xbar stacked,
		cycle list name	= prcl-stacked, 
		bar width       = .35cm,
		xlabel          = {xlabel},
		ylabel          = {opt. ylabel},
		symbolic y coords= {y4,y3,y2,y1},
		ytick           = {y4,y3,y2,y1},
		xtick           = {0,2,...,10},	
		xmin=0, xmax=8,]
		\pgfplotsinvokeforeach{1,...,5}{ 
			\addplot+ table[x=x#1, y=y] {data/stacked.txt}; 
		}
		\end{axis}
		\end{tikzpicture} 
	\end{minipage}  
\end{figure}
\clearpage

%\subsection{Legend} 
\index{Decorations!rotated labels} 
\begin{figure}[h!]
	\begin{minipage}{0.6\textwidth}
		
		\begin{mintedtexbox}[/examples/marks.tex] 
			\inputminted{TeX}{lst/marks_quality1.txt}
		\end{mintedtexbox}
	\end{minipage}	
	\hfill
	\begin{tikzpicture}
	\begin{axis}[	 
	pretty line marks, 
	width=.3\textwidth,
	pr-labels-rotated,	
	symbolic x coords= {Germany,France,England,Saarland,Spain,Finland,USA},	
	legend entries  = {life quality, food quality}, 
	legend columns  = 2,
	legend style    = {
		nodes   = {transform shape}, 
		at		= {(.5,1)}, anchor=north,	
		%% leave space between legend entries:
		/tikz/every even column/.append style={column sep=.3cm},
	}, 
	ymax = 15, ymin = 0
	]
	\addplot table[x=x, y =y1] {data/countries.tsv};
	\addplot table[x=x, y =y2] {data/countries.tsv};
	\end{axis}
	\end{tikzpicture}  
\end{figure}
