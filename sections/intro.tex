\section{Introduction}
This package provides tools for creating easy-to-read plots in Latex. %, and easy-to-read Latex code for plotting. 
 To use it, include 
\begin{minted}{TeX}
    \usepackage{prettyplots} 
\end{minted}
in your preamble.  Then, you can use it together with the \texttt{pgfplots} package, 

\begin{minted}{TeX}
    \usepackage{tikz} 
    \usepackage{pgfplots} 
    \pgfplotsset{compat=1.18}
\end{minted}
where it is recommended to specify a version using \texttt{compat} to avoid backward compatibility issues with \texttt{pgfplots} features. It will be useful to have some basic familiarity with \texttt{pgfplots} \footnote{\texttt{https://pgfplots.sourceforge.net/pgfplots.pdf}}.  

\subsection*{Basic Usage}  
Let's say you want to create a line plot from a dataset of the following shape, 
\begin{figure}[h!]
	\begin{minipage}{0.6\textwidth} 
		\begin{mintedtxtbox}[/data/line.tsv]
\begin{minted}{TeX}
%Metric    Method 
x          y1
0          9 
10         9 
20         8.5
...
\end{minted}
		\end{mintedtxtbox}
	\end{minipage}
\end{figure}

Using \texttt{pgfplots}, you can generate a line plot as follows. 
\begin{figure}[h!]
	\begin{minipage}{0.6\textwidth}  
\begin{tcolorbox}
\begin{minted}{TeX}  
\begin{tikzpicture}  
    \begin{axis}[] 
    \addplot table [x=x, y=y1] {data/line.tsv};
    \end{axis} 
\end{tikzpicture}
\end{minted}
\end{tcolorbox}  
		\end{minipage}
		\begin{minipage}{0.4\textwidth}   
	\begin{tikzpicture}  
	\begin{axis}[   width=\textwidth
		] 
		\addplot table [x=x, y=y1] {data/line.tsv};
	\end{axis} 
\end{tikzpicture} 
	\end{minipage}
	\end{figure} 
\clearpage

\index{Plot Types!line!basic}
To turn this into a pretty plot, add the \texttt{pretty line} command to the pgf axis keys, 
\begin{figure}[h!]
	\begin{minipage}{0.6\textwidth}  
	\begin{tcolorbox}
		\begin{minted}{TeX}   
%preamble
\usepackage{prettyplots} 
%
\begin{tikzpicture}  
    \begin{axis}[pretty line] 
    \addplot table [x=x, y=y1] {data/line.tsv};
    \end{axis} 
\end{tikzpicture} 
		\end{minted}
	\end{tcolorbox}  
\end{minipage}
\begin{minipage}{0.4\textwidth}   
\begin{tikzpicture}  
    \begin{axis}[ 
    pretty line,     width=\textwidth
			] 
			\addplot table [x=x, y=y1] {data/line.tsv};
		\end{axis} 
	\end{tikzpicture} 
\end{minipage}
\end{figure} 


\subsection*{Basic Customization}  
You can keep using the common \texttt{pgfplots} keys to change the appearance of plots. The two main ways to do this are  passing keys to \texttt{axis} to change the global appearance of the axis environment, %axis appearance, labels, or predefined colors; 
or passing  keys to \texttt{addplot} to change the appearance of specific lines or plots. 

\begin{itemize}
	\item \texttt{axis} keys: There are many axis keys you can find in \texttt{pgfplots}. For example, you can set colors and line style to iterate through using a \texttt{cycle list}, such as  \texttt{cycle list=pr-line} which is set by default. 
	\item \texttt{addplot} keys:  arguments you pass to  \texttt{addplot}  change the appearance of each plot that you add to the axis. This overwrites any style defined in the global axis environment, so use \texttt{addplot+} if you want to preserve the default style defined in the axis environment.
\end{itemize}

For example, we can add a line using default style, another with markers, and another that keeps colors  unchanged from the default cycle list \texttt{pr-colors}. 

\index{Plot Types!line!with marks}
\begin{figure}[h!] 
	\hypertarget{mark}{}	
	\begin{minipage}{0.6\textwidth} 
		%\begin{mintedtexbox}[/examples/line-variations.tex]
		\begin{tcolorbox} 
			\begin{minted}{TeX} 
\begin{tikzpicture} 
   \begin{axis}[ 
        pretty line, 
        cycle list name = pr-colors,
        legend style = {anchor=north west}
    ]  
    \addplot table [x=x, y=y1] {data/line.tsv}; 
    \addlegendentry{default};
    \addplot[mark=*, mark size=2pt] 
    table [x=x, y=y2] {data/line.tsv}; 
    \addlegendentry{\texttt{\tbs addplot}};
    \addplot+[mark=diamond*, mark size=2pt] 
    table [x=x, y=y3] {data/line.tsv}; 
    \addlegendentry{\texttt{\tbs addplot+}};
    \end{axis} 
\end{tikzpicture} 
			\end{minted}
	%	\end{mintedtexbox}
		\end{tcolorbox}
	\end{minipage}
	\begin{minipage}{0.4\textwidth} 
\begin{tikzpicture} 
		\begin{axis}[ pretty line, 
		legend style = {anchor=north west},
		cycle list name=pr-colors, 
		width=\textwidth 
		]  
		\addplot table [x=x, y=y1] {data/line.tsv}; 
		\addlegendentry{default};
		\addplot[mark=*, mark size=2pt] table [x=x, y=y2] {data/line.tsv}; 
		\addlegendentry{\texttt{\tbs addplot}};
		\addplot+[mark=diamond*, mark size=2pt] table [x=x, y=y3] {data/line.tsv}; 
		\addlegendentry{\texttt{\tbs addplot+}};
	\end{axis} 
\end{tikzpicture} 
	\end{minipage} 	 
\end{figure}   

\clearpage 
Here is a larger example using some common \texttt{pgfplots} axis keys to customize sizes, positioning, and labels. 

\begin{figure}[h!]
	\begin{minipage}{0.6\textwidth}  
	\begin{tcolorbox}
		\begin{minted}{TeX}  
\begin{tikzpicture}  
    \begin{axis}[ 
        pretty line,  
        width = \textwidth,
        xlabel = {xname}, 
        ylabel = {yname},
        ytick = {0,2,4,6,8,10},
        ymin = 0, ymax = 10,
        legend style = {anchor=south east},
        legend entries = 
             {Method 1, Method 2, Method 3},
        smooth,
        line width = 1.5pt,
    ] 
    \foreach \i in {1,2,3}   {
        \addplot table [x=x, y=y\i] {data/line.tsv}; 
    }
\end{axis} 
\end{tikzpicture} 
		\end{minted}
	\end{tcolorbox}  
\end{minipage}
\begin{minipage}{0.4\textwidth}   
	\begin{tikzpicture}   
		\begin{axis}[ 
			pretty line,  
			width = \textwidth,
			legend entries = {Method 1, Method 2, Method 3},    
			legend style={ anchor=south east},
			ytick = {0,2,4,6,8,10},
			ymin = 0, ymax = 10,
			xlabel = {xname}, 
			ylabel = {yname},  
			smooth, 
			line width = 1.5pt
			] 
			\foreach \i in {1,2,3}   {
				\addplot table [x=x, y=y\i] {data/line.tsv}; 
			}
		\end{axis} 
	\end{tikzpicture} 
\end{minipage}
\end{figure}
 To create a different type of plot, modify the command depending on the type using \texttt{pretty  $\langle$\emph{plotname}$\rangle$}.  %todo Figure \ref{fig:ex:box_bar} gives an overview on the plot types. 

%\subsection*{Usage}   Overall, you can use the library for the following purposes, 
%\begin{itemize}
%	\item to create easier-to-read plots using the \texttt{pretty  $\langle$\emph{plotname}$\rangle$} axis environments;
%	\item to use some of the example plots and color schemes as a starting point; or 
%	\item to write easier-to-read Latex code using macros in case you have repetitive use of the same plots.
%\end{itemize}

 The main axis environments are shown in \sectref{sec:plots}, color schemes in \sectref{sec:style}, and macros in  \sectref{sec:macro}. 
